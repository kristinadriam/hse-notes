\subsection{Матричная запись линейного оператора}

\begin{definition}
    $A: \R^m\rightarrow \R^n$. $e_i=\begin{pmatrix} 0 \\ ... \\ 1 \\ ... \\ 0\end{pmatrix}i$

    $A(e_i)\in \R^n$, $A(e_i)=A_i=\begin{pmatrix} a_{i1} \\ a_{i2} \\ ... \\ a_{in}\end{pmatrix}$, $x=\begin{pmatrix} x_{i1} \\ x_{i2} \\ ... \\ x_{in}\end{pmatrix}$

    $A(x)=A(\sum\limits_{i=1}^m x_ie_i)=\sum\limits_{i=1}^m x_iA(e_i)=\sum\limits_{i=1}^m x_iA_i$

    $\begin{pmatrix} a_{11} & a_{21} & ... & a_{m1} 
    \\
    a_{12} & a_{22} & ... & a_{m2} \\
    ... & ... & ... & ... \\
    a_{1n} & a_{2n} & ... & a_{mn} 
    \end{pmatrix}\begin{pmatrix} x_{i1} \\ x_{i2} \\ ... \\ x_{in}\end{pmatrix}=\begin{pmatrix} x_1 a_{11} + x_2 a_{21} + ... + x_ma_{m1} 
    \\
    x_1 a_{12} + x_2 a_{22} + ... + x_m a_{m2} \\
    ... \\
    x_1 a_{1n} + x_2 a_{2n} + ... + x_m a_{mn} 
    \end{pmatrix}$
\end{definition}

\begin{definition}
    $X$ и $Y$ – нормированные пространсnва, $A:X\rightarrow Y$ – линейный оператор. $\|A\|:=\sup\limits_{\|x\|_X\leq 1} \|A_X\|_Y$ – \textit{норма}. Если $\|A\|<+\infty$, то $A$ – \textit{ограниченный оператор}.
\end{definition}

\begin{remark}

    \begin{enumerate}
        \item[]
        \item $A(B_1(0))\subset B_{\|A\|}(0)$.
        \item Ограниченный оператор $\neq$ ограниченное отображение. Более того, линейный оператор, являющийся ограниченным отображением – тождественный ноль.

        Если $A_{x_0}\neq 0$, то $\|A(tx_0)\|=\|tAx_0\|=|t|\cdot \|Ax_0\|\underset{t\rightarrow+\infty}{\rightarrow }+\infty$

        \item Бывают неограниченные операторы.

        $X:=\{(x_1, x_2, ...)\mid $ лишь конечное число $x_n\neq 0\}$, $\|(x_n)\|:=\max\limits_{n\in \N}|x_n|$, $Ax:=\sum\limits_{i=1}^\infty x_i$.
    \end{enumerate}
    
\end{remark}

\begin{statement}
    \textbf{ Свойства нормы:}

    \begin{enumerate}
        \item $\|A+B\|\leq \|A\| + \|B\|$.
        \begin{proof}
            $\|A+B\|=\sup\limits_{\|x\|\leq 1} \|(A+B)x\|=\sup\limits_{\|x\|\leq 1} \|Ax+Bx\|\leq \sup\limits_{\|x\|\leq 1} (\|Ax\|+\|Bx\|)\leq \sup\limits_{\|x\|\leq 1} \|Ax\|+\sup\limits_{\|x\|\leq 1} \|Bx\|=\|A\|+\|B\|$.
        \end{proof}
        \item $\|\lambda A\|= |\lambda|\|A\|$ $\lambda\in \R.$
        \begin{proof}
            $\|\lambda A\|=\sup\limits_{\|x\|\leq 1} \|\lambda Ax\|=\sup\limits_{\|x\|\leq 1}|\lambda|\|Ax\| =|\lambda|\|A\|$
        \end{proof}
        \item Если $\|A\|=0$, то $A\equiv0$.
        \begin{proof}
            Если $\|A\|=0$, то $\|Ax\|=0$ $\forall x\in X: \|x\|\leq 1\Rightarrow Ax=0\ \forall x\in X: \|x\|\leq 1.$

            Возьмем $y_{\neq 0}\in X$. $\frac{y}{\|y\|}$ – единичный вектор. Тогда $\underset{=\frac{1}{\|y\|}Ay}{A(\frac{y}{\|y\|})}=0\Rightarrow Ay=0$
        \end{proof}
        \item $\|\cdot\|$ – норма на векторном пространтсве ограниченных линейных операторов.
    \end{enumerate}
\end{statement}

\begin{theorem}
    $\sup\limits_{\|x\|\leq 1}\|Ax\|= \sup\limits_{\|x\|< 1}\|Ax\|=\sup\limits_{\|x\|=1}\|Ax\|=\sup\limits_{x\neq 0}\frac{\|Ax\|}{\|x\|}=\inf \{C\in \R\mid \|Ax\|\leq C\|x\|\ \forall x\in X\}$

    $N_1=N_2=N_3=N_4=N_5$
\end{theorem}

\begin{proof}
    

\begin{enumerate}
    \item[]

    \item[$\circ$] $N_1\geq N_2$ и $N_1\geq N_3$ очевидно (множество больше $\Rightarrow \sup$ больше).
    
    \item[$\circ$] $N_3=N_4$ 
    
    $\frac{\|Ax\|}{\|x\|}=\frac{1}{\|x\|}\cdot \|Ax\|=\|A(\frac{x}{\|x\|})\|$

    $\sup\limits_{x\neq 0}\frac{\|Ax\|}{\|x\|}=\sup\limits_{x\neq 0}\|A(\frac{x}{\|x\|})\|=\sup\limits_{\|x\|=1}\|Ax\|$

    \item[$\circ$] $N_4=N_5$

    $\sup\limits_{x\neq 0} \frac{\|Ax\|}{\|x\|}$ – наименьшая верхняя граница, то есть наименьшее $C\in R$, для которого $\frac{\|Ax\|}{\|x\|}\leq C\Leftrightarrow \|Ax\|\leq C\|x\|$

    \item[$\circ$] $N_2\geq N_1$

    Возьмем $x\in X: \|x\|\leq 1$.  Тогда $\|(1-\varepsilon)x\|\leq 1-\varepsilon < 1$

    $(1-\varepsilon)\|Ax\|=\|A(1-\varepsilon)x\|\leq N_2\Rightarrow (1-\varepsilon)N_1\leq N_2$ и устремим $\varepsilon$ к 0. 

    \item[$\circ$] $N_3\geq N_1$

    Возьмем $x\in X: \|x\|\leq 1$ и $x\neq 0$. Тогда $y=\frac{x}{\|x\|}$ – единичный вектор. 

    $N_3\geq \|Ay\|=\|A(\frac{x}{\|x\|})\|=\frac{1}{\|x\|}\|Ax\|\Rightarrow \|Ax\|\leq \|x\|\cdot N_3\leq N_3\Rightarrow N_1\leq N_3$
\end{enumerate}
\end{proof}

\begin{corollary}
\begin{enumerate}
    \item[]
    \item $\|Ax\|\leq \|A\|\cdot \|x\|$
    \begin{proof}
        Это $N_4:$ $\frac{\|Ax\|}{\|x\|}\leq \|A\|$
    \end{proof}

    \item $\|AB\|\leq \|A\|\cdot \|B\|$
    \begin{proof}
        $\|AB\|=\sup\limits_{\|x\|\leq 1}\|A(Bx)\|\leq\sup\limits_{\|x\|\leq 1}\|A\|\cdot \|Bx\|=\|A\|\cdot \sup\limits_{\|x\|\leq 1}\|Bx\|=\|A\|\cdot \|B\|$
    \end{proof}
\end{enumerate}
\end{corollary}

\begin{theorem}
    Пусть $A:X\rightarrow Y$ – линейный оператор. Тогда равносильны:
    \begin{enumerate}
        \item $A$ – ограниченный оператор.
        \item $A$ непрерывен в нуле.
        \item $A$ непрерывен во всех точках.
        \item $A$ равномерно непрерывен.
    \end{enumerate}
\end{theorem}

\begin{proof}
    $4\Rightarrow 3\Rightarrow 2$ очевидны.

    $1\Rightarrow 4:$ $\|Ax-Ay\|=\|A(x-y)\|\leq \|A\|\cdot \|x-y\|$ и возьмем $\delta=\frac{\varepsilon}{\|A\|}$. Если $\|x-y\|<\delta$, то $\|Ax-Ay\|<\varepsilon$.

    $2\Rightarrow 1:$ Возьмем $\varepsilon =1$ и  $\delta>0$ из определения непрерывности в нуле. 

    $\forall x\in X\ \|x\|<\delta \Rightarrow \|Ax\|\leq\varepsilon =1$. Тогда $\|A\|\leq \frac{\varepsilon}{\delta}$:

    $\|x\|<\delta\Rightarrow$ если $\|y\|<1$, то $\|\delta y\|<\delta$ и $\|A(\delta y)\|=\delta\|A y\|<\varepsilon\Rightarrow \|A\|<\frac{\varepsilon}{\delta}$.
\end{proof}

\begin{theorem}
    $A:\R^n\rightarrow R^m$ – линейный оператор. Тогда $\|A\|^2\leq \sum\limits_{i=1}^m\sum\limits_{k=1}^n a^2_{ik}$. В частности $A$ – ограниченный оператор.
\end{theorem}

\begin{proof}
    $\|Ax\|^2=\sum\limits_{i=1}^m(\sum\limits_{k=1}^n a_{ik}x_k)^2\overset{\text{К.-Б.}}{\leq}\sum\limits_{i=1}^m(\sum\limits_{k=1}^n a^2_{ik}\cdot \sum\limits_{k=1}^n x_k^2)=\underbrace{\sum\limits_{k=1}^n x^2_k}_{=\|x\|^2}\cdot \sum\limits_{i=1}^m\sum\limits_{k=1}^n a^2_{ik}$

    $\Rightarrow \|A\|^2\leq \sqrt{\sum\limits_{i=1}^m\sum\limits_{k=1}^n a^2_{ik}}$
\end{proof}

\newpage
\section{Ряды}

\subsection{Ряды в нормированном пространстве}

\begin{definition}
    Пусть $X$ – нормированное пространство. Тогда $\sum\limits_{n=1}^\infty x_n$, где $x_n\in X$ – \textit{ряд}.

    \textit{Частичная сумма ряда} $\sum\limits_{k=1}^nx_k:=S_n$.

    Если существует $\lim S_n$, то он называется \textit{суммой ряда}.

    Ряд сходится, если $\lim S_n$ существует.
\end{definition}

\begin{remark} Для числовых рядов сходимость означает, что $\lim S_n\in \R$.
\end{remark}

\begin{theorem}
    \textbf{{Необходимое} условие сходимости}

    Если ряд сходится, то $\lim x_n=0$.
\end{theorem}

\begin{proof}
    $\lim x_n=\lim (S_n-S_{n-1})=\lim S_n-\lim S_{n-1}=0$
\end{proof}

\begin{statement}
    \textbf{Свойства:}
    \begin{enumerate}
        \item Если $\sum\limits_{n=1}^\infty x_n$ и $\sum\limits_{n=1}^\infty y_n$ сходятся, то $\forall \alpha, \beta\in \R$ $\sum\limits_{n=1}^\infty (\alpha x_n+\beta y_n)$ сходится и $\sum\limits_{n=1}^\infty (\alpha x_n+\beta y_n)=\alpha \sum\limits_{n=1}^\infty x_n+\beta \sum\limits_{n=1}^\infty y_n$.
        
        \item Если ряд $\sum\limits_{n=1}^\infty x_n$ сходится, то расстановка скобок не меняет сумму.

        $(x_1+x_2+x_3)_{=S_3}+x_{4_{=S_4}}+(x_5+x_6)_{=S_6}+...$

        \begin{remark}
            Расстановка скобок – выбор подпоследовательности в последовательности частичных сумм.
        \end{remark}
    \end{enumerate}
\end{statement}

\begin{theorem}
    \textbf{Критерий Коши}

    Пусть $X$ – полное нормированное пространство. 
    
    Тогда $\sum\limits_{n=1}^\infty x_n$ – сходится $\Leftrightarrow \forall \varepsilon >0\ \exists N: \forall n>m\geq N\ \|\sum\limits_{k=m+1}^n x_k\|<\varepsilon$. 
\end{theorem}

\begin{proof}
    $\sum\limits_{n=1}^\infty x_n$ – сходится $\Leftrightarrow S_n=\sum\limits_{k=1}^n x_k$ имеет предел $\Leftrightarrow S_n$ – фундаментальная последовательность $\Leftrightarrow \varepsilon >0\ \exists N: \forall n>m\geq N\ \|S_n-S_m\|=\|\sum\limits_{k=m+1}^n x_k\|<\varepsilon $ 
\end{proof}

\begin{definition}
    Ряд $\sum\limits_{n=1}^\infty x_n$ \textit{сходится абсолютно}, если $\sum\limits_{n=1}^\infty \|x_n\|$ сходится (для числовых рядов это означает, что $\sum\limits_{n=1}^\infty |x_n|$ сходится).
\end{definition}

\begin{theorem}
    Пусть $X$ – полное нормированное пространство. Если $\sum\limits_{n=1}^\infty x_n$ абсолютно сходится, то $\sum\limits_{n=1}^\infty x_n$ сходится.
\end{theorem}

\begin{proof}
    Критерий Коши: $\sum\limits_{n=1}^\infty x_n$ абсолютно сходится $\Rightarrow$ $\sum\limits_{n=1}^\infty \|x_n\|$ абсолютно сходится 
    
    $\Rightarrow\forall \varepsilon >0\ \exists N: \forall n>m\geq N\ \underbrace{\sum\limits_{k=m+1}^\infty\| x_k\|}_{\geq \|\sum\limits_{k=m+1}^\infty x_k\|}<\varepsilon\Rightarrow \sum\limits_{n=1}^\infty x_n $ сходится.
\end{proof}

\begin{theorem}
    \textbf{Группировка членов ряда}

    \begin{enumerate}
        \item Если каждая группа содержит не более $M$ слагаемых и $\lim x_n=0$, то из сходимости сгрупированного ряда следует сходимость исходного.

        \begin{proof}
            По условию $\lim S_{n_k}=S$ и $n_{k+1}-n_k\leq M$.

            Возьмем какое-то $n$. Тогда $n_k\leq n<n_{k+1}$ для какого-то $k$.

            $S_n=S_{n_k}+x_{n_k+1}+x_{n_k+2}+...+x_n$.

            $\|S_n-S\|=\|S_{n_k}-S+x_{n_k+1}+x_{n_k+2}+...+x_n\|\leq \underbrace{\|S_{n_k}-S\|}_{<\varepsilon}+\underbrace{\|x_{n_k+1}\|}_{<\varepsilon}+...+\underbrace{\|x_n\|}_{<\varepsilon}<\varepsilon\cdot (M+1)$

            $\exists K: \forall k\geq K\ \|S_{n_k}-S\|<\varepsilon$

            $\exists N: \forall m\geq N\ \|x_m\|<\varepsilon$
        \end{proof}

        \item Для числовых рядов: если члены в каждой группе одного знака, то из сходимости сгрупированного ряда следует сходимость исходного.

        \begin{proof}
            Возьмем какое-то $n$: $n_k\leq n<n_{k+1}$. Пусть в блоке все слагаемые неотрицательные.


            $\begin{matrix*}[l]
                S_n=S_{n_k}+x_{n_k+1}+x_{n_k+2}+...+x_n\geq S_{n_k} \\
                S_n=S_{n_{k+1}}-x_{n_k+1}-x_{n_k+2}-...-x_n\leq S_{n_{k+1}}\text{ (вычитаем неотрицательные числа)}
            \end{matrix*}\Rightarrow$
            
            $\Rightarrow\underset{\rightarrow S}{S_{n_k}}\leq S_n\leq \underset{\rightarrow S}{S_{n_{k+1}}}$ или $S_{n_{k+1}}\leq S_n\leq S_{n_{k}}$ (если в блоке все слагаемые $\leq 0$)$\Rightarrow \lim S_n=S$.
        \end{proof}
    \end{enumerate}
\end{theorem}


\subsection{Знакопостоянные ряды}

\begin{theorem}
    Пусть $a_n\geq 0$. Тогда ряд $\sum\limits_{n=1}^\infty a_n$ сходится $\Leftrightarrow$ последовательность частичных сумм ограничена.
\end{theorem}

\begin{proof}
    $S_n=\sum\limits_{k=1}^n a_k$ и $S_{n+1}=S_n+a_{n + 1}\geq S_n\Rightarrow S_n$ монотонно возрастает

    $\sum a_n$ – сходится $\Leftrightarrow S_n$ имеет конечный предел $\Leftrightarrow S_n$ ограничена (так как монотонна).
\end{proof}

\begin{theorem}
    \textbf{Признак сравнения}

    Пусть $0\leq a_n\leq b_n$. Тогда:

    \begin{enumerate}
        \item Если ряд $\sum\limits_{n=1}^\infty b_n$ сходится, то и $\sum\limits_{n=1}^\infty a_n$ сходится.

        \item Если ряд $\sum\limits_{n=1}^\infty a_n$ расходится, то и $\sum\limits_{n=1}^\infty b_n$ расходится.
    \end{enumerate}
\end{theorem}

\begin{proof}
    Второй пункт – отрицание первого, поэтому докажем только первый.

    $A_n:=\sum\limits_{k=1}^n a_k$, $B_n:=\sum\limits_{k=1}^n b_k\Rightarrow A_n\leq B_n$

     $\sum\limits_{n=1}^\infty b_n$ сходится $\Rightarrow B_n$ – ограниченная последовательность $\Rightarrow A_n$ – ограниченная последовательность $\Rightarrow$ $\sum\limits_{n=1}^\infty a_n$ сходится.
\end{proof}

\begin{corollary}
    \begin{enumerate}
        \item[]
        \item Если $a_n, b_n\geq 0$, $a_n=\mathcal{O}(b_n)$ и $\sum\limits_{n=1}^\infty b_n$ сходится, то и $\sum\limits_{n=1}^\infty a_n$ сходится.

        \begin{proof}
            $0\leq a_n \leq C\cdot b_n$
        \end{proof}
        
        \item Если $a_n, b_n\geq 0$ и $a_n\sim b_n$, то ряды $\sum\limits_{n=1}^\infty a_n$ и $\sum\limits_{n=1}^\infty b_n$ ведут себя одинаково.

        \begin{proof}
            $a_n\sim b_n\Rightarrow \frac{1}{2}a_n\leq b_n\leq 2 b_n$ при больших $n$,
        \end{proof}
    \end{enumerate}
\end{corollary}

\begin{theorem}
    \textbf{Признак Коши}

    Пусть $a_n\geq 0$.

    \begin{enumerate}
        \item  Если $\sqrt[n]{a_n}\leq q< 1$ при больших $n$, то $\sum\limits_{n=1}^\infty a_n$ сходится.
        \begin{proof}
            Поменяем начальные $a$-шки так, чтобы $\sqrt[n]{a_n}\leq q$ для всех $n\Rightarrow a_n\leq q^n$

            $\sum\limits_{n=1}^\infty q^n$ – сходящийся (геометрическая прогрессия с $q < 1$) $\Rightarrow $ признак сравнения.
        \end{proof}

        \item Если $\sqrt[n_k]{a_{n_k}}\geq 1$ при больших $n$, то $\sum\limits_{n=1}^\infty a_n$ расходится.

        \begin{proof}
            Сколь угодно далеко есть члены ряда больше 1 $\Rightarrow$ нет необходимого условия сходимости.
        \end{proof}
        \item Пусть $q^*=\overline{\lim}\sqrt[n]{a_n}$. 

        Если $q^*<1$, то ряд сходится.
        
        Если $q^*>1$, то ряд расходится.
        
        \begin{remark}
            Если $q^*=1$, то ряд может сходиться, а может и расходиться.
        \end{remark}

        \begin{proof}
            Если $q^*>1$ и $\overline{\lim}$ – какой-то частичный предел $\Rightarrow$ найдется последовательность $n_k$: $\lim\sqrt[n_k]{a_{n_k}}>1\Rightarrow$ при больших $k$ $\sqrt[n_k]{a_{n_k}}>1\Rightarrow$ ряд расходится по второму пункту.

            Если $q^*<1$ и $q^*=\overline{\lim}=\lim\underbrace{\sup\limits_{k\geq n} \sqrt[n_k]{a_{n_k}}}_{:=b_n}$.

            При больших $n$ $\sqrt[n]{a_n}\leq b_n<q\Rightarrow \sqrt[n]{a_n}<q\Rightarrow$ по первому пункту ряд сходится.
        \end{proof}
    \end{enumerate}
\end{theorem}


\begin{theorem}
    \textbf{Признак Даламбера}

    Пусть $a_n> 0$.

    \begin{enumerate}
        \item Если $\frac{a_{n+1}}{a_n}\leq d<1$ при больших $n$, то ряд сходится.
        \begin{proof}
            Подправим начало так, чтобы неравенство было верным для всех $n$.      
            
            $a_{n+1}\leq d\cdot a_n\leq d^2\cdot a_{n-1}\leq...\leq d^n\cdot a_1\Rightarrow a_n=\mathcal{O}(d^{n-1})$

            $\sum\limits_{n=1}^\infty d^{n-1}$ – сходится, так как $d<1$.
        \end{proof}

        \item Если $\frac{a_{n+1}}{a_n}\geq$ при больших $n$, то ряд расходится.

        \begin{proof}
            $a_n\leq a_{n+1}$ последовательность положительна и возрастает $\Rightarrow$ нет стремления к 0 $\Rightarrow$ ряд расходится.
        \end{proof}
        
        \item Пусть $d^*=\lim \frac{a_{n+1}}{a_n}$.

        Если $d^*<1$, то ряд сходится.

        Если $d^*>1$, то ряд расходится.
        
        \begin{remark}
            Если $d^*=1$, то ряд может сходиться, а может и расходиться.
        \end{remark}

        \begin{proof}
            Пусть $d^*<1$. Тогда $\lim \frac{a_{n+1}}{a_n}=d^*<1\Rightarrow$ при больших $n$ $\frac{a_{n+1}}{a_n}<d\Rightarrow$ по 1 пункту ряд сходится.

            Пусть $d^*>1$. Тогда $\lim \frac{a_{n+1}}{a_n}=d^*>1\Rightarrow$ при больших $n$ $\frac{a_{n+1}}{a_n}>1\Rightarrow$ по 2 пункту ряд расходится.
        \end{proof}
    \end{enumerate}
\end{theorem}

\begin{example}
    \begin{enumerate}
        \item[]
        \item $\sum\limits_{n=1}^\infty\frac{1}{n}$ расходится: $\lim \sqrt[n]{\frac{1}{n}}=1$ и $\lim\frac{\frac{1}{n+1}}{\frac{1}{n}}=1$
        \item $\sum\limits_{n=1}^\infty\frac{1}{n(n+1)}$ сходится: $\lim \sqrt[n]{\frac{1}{n(n+1)}}=1$ и $\lim\frac{\frac{1}{(n+1)(n+2)}}{\frac{1}{n(n+1)}}=1$
    \end{enumerate}
\end{example}

\begin{theorem}
    Пусть $a_n>0$. Если $d^*=\lim\frac{a_{n+1}}{a_n}$, то $\lim\sqrt[n]{a_n}=d^*$.
\end{theorem}

\begin{proof}
    $\ln \sqrt[n]{a_n}=\frac{\ln a_n}{n}$ по теорему Штольца надо найти $\lim\frac{\ln a_{n+1}-\ln a_n}{(n+1)-n}=\lim \ln\frac{a_{n+1}}{a_n}=\ln d^*\Rightarrow \lim \ln \sqrt[n]{a_n}=\ln d^*\Rightarrow \lim\sqrt[n]{a_n}=d^*$.
\end{proof}

