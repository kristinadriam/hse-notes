\documentclass[10pt, oneside, dvipsnames]{extarticle}
% шрифтики
\usepackage{amsmath}
\usepackage{bbm}

\linespread{1.35}
%вставка изображений и картинок всяких

\usepackage{graphicx}
\newcommand{\incfig}[1]{%
    \def\svgscale{1.5}
    \import{./figures/}{#1.pdf_tex}
}
\graphicspath{{pictures/}}
\DeclareGraphicsExtensions{.pdf,.png,.jpg, .jpeg, .tex}

\usepackage{booktabs} % для таблиц
\usepackage{enumitem} % для списков

\usepackage [italicdiff]{physics}


% Шрифты 
\usepackage[T2A]{fontenc}
\usepackage[english,russian]{babel}


\usepackage{titling} % для \maketitle
\usepackage{textcomp}% специальные символы в тексте

\usepackage{mathtext}
\usepackage{mathtools}
\usepackage{amsmath,amsfonts,amssymb,amsthm,mathtools} % математика
\usepackage{icomma} % умная запятая
\usepackage{import} %  импортирование


\usepackage{pdfpages} % мультри-пдф страницы
\usepackage{transparent} % что-то про цвета
\usepackage{xcolor} % цвета

\usepackage{caption} % комментарии к figure
\usepackage{epigraph} % эпиграфы

\usepackage{comment} % удобные комментарии
\usepackage{xfrac} % дроби
\usepackage{moresize} % все размеры шрифтов
\usepackage{dsfont} % mathbb для всего


% Окружения для математики:

\newtheorem*{corollary}{Следствие}
\newtheorem{theorem}{Теорема}[subsection]
\theoremstyle{definition}
\newtheorem{statement}{Утверждение}[subsection]
\newtheorem{definition}{Определение}[subsection]
\newtheorem{designation}{Обозначение}
\newtheorem*{proper}{Свойства}

\newtheorem*{example}{Пример}
\newtheorem{antiexample}{Антиример}
\newtheorem{lemma}{Лемма}[subsection]
\theoremstyle{remark}
\newtheorem*{remark}{Замечание}
\newtheorem*{exercise}{Упражнение}

% Настройки счетчиков:

%\numberwithin{se%ction} % Number equations within sections (i.e. 1.1, 1.2, 2.1, 2.2 instead of 1, 2, 3, 4)
%\numberwithin{f%igure}{section} % Number figures within sections (i.e. 1.1, 1.2, 2.1, 2.2 instead of 1, 2, 3, 4)
%\numberwithin{table}{section} % Number tables within sections (i.e. 1.1, 1.2, 2.1, 2.2 instead of 1, 2, 3, 4)

% Геометрия файла:

\usepackage{geometry}
\usepackage{indentfirst}

%\setlist{noitemsep} % No spacing between list items

\geometry{left=1.5cm,right=1.5cm,top=2.5cm,bottom=2cm, a4paper}
\setlength{\parindent}{0cm}

% Счётчики разделов:


%\sectionfont{\vspace{6pt}\centering\normalfont\scshape} % \section{} styling
%\subsectionfont{\normalfont\bfseries} % \subsection{} styling
%\subsubsectionfont{\normalfont\itshape} % \subsubsection{} styling
%\paragraphfont{\normalfont\scshape} % \paragraph{} styling

\newcommand{\RNumb}[1]{\uppercase\expandafter{\romannumeral #1\relax}}

\renewcommand\thesection{\arabic{section}.}
\renewcommand\thesubsection{\thesection\arabic{subsection}}
\renewcommand\thesubsubsection{\RNumb{\arabic{subsubsection}}}
\renewcommand{\bf}{\textbf}

% Колонтитулы

\usepackage{fancyhdr}
\pagestyle{fancy}
\fancyhf{} % clear all fields
\fancyhead[L]{\rightmark}
\fancyhead[R]{\thepage}

\renewcommand{\sectionmark}[1]{%
  \markright{\thesection\ #1}}%
\setlength{\headheight}{17.0pt}
\addtolength{\topmargin}{-2.49998pt}



% Операторы:

\DeclareMathOperator{\ord}{ord}
\DeclareMathOperator{\ld}{ld}
\DeclareMathOperator{\esssup}{ess\ sup}
\DeclareMathOperator{\supp}{supp}
\DeclareMathOperator{\exi}{exi}
\DeclareMathOperator{\num}{num}
\DeclareMathOperator{\card}{Card}
\DeclareMathOperator{\sk}{sk}
\DeclareMathOperator{\den}{den}
\DeclareMathOperator{\ran}{ran}
\DeclareMathOperator{\dom}{dom}
\DeclareMathOperator{\diam}{diam}
\DeclareMathOperator{\sign}{sign}
\DeclareMathOperator{\Int}{Int}
\DeclareMathOperator{\Br}{}

\DeclareMathOperator{\RelInt}{RelInt}
\DeclareMathOperator{\Cl}{Cl}
\DeclareMathOperator{\pr}{pr}
\DeclareMathOperator{\area}{area}
\DeclareMathOperator{\Af}{Aff}
\renewcommand{\Im}{\mathop{\mathrm{Im}}\nolimits}
\DeclareMathOperator{\Conv}{conv}
\DeclareMathOperator{\rad}{rad}
\DeclareMathOperator{\diag}{diag}
\DeclareMathOperator{\Fr}{Fr}
\DeclareMathOperator{\Span}{span}
\DeclareMathOperator{\Map}{Map}
\DeclareMathOperator{\osc}{osc}
\DeclareMathOperator{\Ker}{Ker}
\DeclarePairedDelimiter\lr{(}{)}
\DeclareRobustCommand{\divby}{%
     \mathrel{\text{\vbox{\baselineskip.65ex\lineskiplimit0pt\hbox{.}\hbox{.}\hbox{.}}}}%
}
\newcommand{\eqdef}{\stackrel{\mathrm{def}}{=}}
\DeclareRobustCommand{\notdivby}{%
     \!\!\not\;\divby%
}
\newcommand{\Mod}[1]{\ (\mathrm{mod}\ #1)}

% Tikz и графика:


\usepackage{pgfplots}
\usepackage{tikz}
\usetikzlibrary{3d,perspective}
\usetikzlibrary{animations}
\usepackage{tikz-cd} % коммутативные диаграммы
\usetikzlibrary{cd}
\pgfplotsset{width=6cm,compat=newest}


%%%% гиперссылки

\usepackage[colorlinks, urlcolor = Blue, filecolor = Blue,citecolor = Blue, linkcolor = Blue]{hyperref}

% Буковы

\newcommand{\N}{\mathbb{N}}			 		
\newcommand{\Z}{\mathbb{Z}}			
\newcommand{\Q}{\mathbb{Q}}		
\newcommand{\R}{\mathbb{R}}	
\let\oldC\C
\renewcommand{\C}{\mathbb{C}} 				
\newcommand{\F}{\mathbb{F}}	
 
\let\oldAA\AA
\renewcommand{\AA}{\mathbb{A}}				
\newcommand{\DD}{\mathbb{D}} 						
\newcommand{\EE}{\mathbb{E}} 			
\newcommand{\HH}{\mathbb{H}}					
\newcommand{\KK}{\mathbb{K}} 					
\newcommand{\OO}{\mathbb{O}} 		
\newcommand{\PP}{\mathbb{P}}			
\let\oldSS\SS		
\renewcommand{\SS}{\mathbb{S}}						
\newcommand{\TT}{\mathbb{T}} 			

\newcommand{\cA}{\mathcal{A}}
\newcommand{\cB}{\mathcal{B}}
\newcommand{\cC}{\mathcal{C}}
\newcommand{\cD}{\mathcal{D}}
\newcommand{\cE}{\mathcal{E}}
\newcommand{\cF}{\mathcal{F}}
\newcommand{\cG}{\mathcal{G}}
\newcommand{\cH}{\mathcal{H}}
\newcommand{\cI}{\mathcal{I}}
\newcommand{\cJ}{\mathcal{J}}
\newcommand{\cK}{\mathcal{K}}
\newcommand{\cL}{\mathcal{L}}
\newcommand{\cM}{\mathcal{M}}
\newcommand{\cN}{\mathcal{N}}
\newcommand{\cO}{\mathcal{O}}
\newcommand{\cQ}{\mathcal{Q}}
\newcommand{\cP}{\mathcal{P}}
\newcommand{\cR}{\mathcal{R}}
\newcommand{\cS}{\mathcal{S}}
\newcommand{\cT}{\mathcal{T}}
\newcommand{\cU}{\mathcal{U}}
\newcommand{\cV}{\mathcal{V}}
\newcommand{\cW}{\mathcal{W}}
\newcommand{\cX}{\mathcal{X}}
\newcommand{\cY}{\mathcal{Y}}
\newcommand{\cZ}{\mathcal{Z}}
			
\newcommand{\rD}{\mathrm{D}}
\newcommand{\rK}{\mathrm{K}}			
\newcommand{\rP}{\mathrm{P}}
\newcommand{\rT}{\mathrm{T}}			

\newcommand{\fA}{\mathfrak{A}}
\newcommand{\fB}{\mathfrak{B}}
\newcommand{\fM}{\mathfrak{M}}
\newcommand{\fL}{\mathfrak{L}}
\newcommand{\fR}{\mathfrak{R}}
\newcommand{\fP}{\mathfrak{P}}
\newcommand{\fC}{\mathfrak{C}}
\newcommand{\fX}{\mathfrak{X}}

\newcommand{\fm}{\mathfrak{m}}
\newcommand{\fp}{\mathfrak{p}}

\newcommand{\Lin}{\text{Lin }}
\newcommand{\rk}{\text{rk }}

\newcommand{\const}{\text{const}}
\newcommand\frightarrow{\scalebox{1}[.4]{$\rightarrow$}}
\newcommand\darrow[1][]{\mathrel{\stackon[1pt]{\stackanchor[1pt]{\frightarrow}{\frightarrow}}{\scriptstyle#1}}}

\usepackage{cleveref}
\begin{document}

\newpage

\begin{definition}
    Пусть $f: E\rightarrow \R^m$, $E\subset \R^n$, $a\in \Int E$. Тогда $f$ \textit{дифференцируема в точке} $a$, если существует линейное отображение $T: \R^n \rightarrow \R^m$, т.ч. 
    
    $f(a + h) = f(a) + Th + o(\|h\|)$ при $h\rightarrow 0$.
\end{definition}

\begin{remark}
    Дифференцируемость в точке $a$ влечет непрерывность в точке $a$.

    $f(a+h)=\underset{\rightarrow f(a)}{f(a)}+\underset{\rightarrow T0=0}{Th}+\underset{\rightarrow 0}{o(\|h\|)}$ при $h \rightarrow 0$
\end{remark}

\begin{theorem}
    Пусть $f: E\rightarrow \R^m$, $E\subset \R^n$, $a\in \Int E$, $f=\begin{pmatrix}
        f_1  \\ \vdots \\ f_m
    \end{pmatrix}$, где $f_1, ..., f_m: E\rightarrow \R$ (\textit{координатные функции}). Тогда $f$ дифференцирема в точке $a\Leftrightarrow f_j$  дифференцируема в точка $a$ $\forall j=1, ..., m$.
\end{theorem}


\begin{corollary}
    Матрица Якоби $f$ – матрица, составленная из градиентов координатных функций: $f'(a)=\begin{pmatrix}
             \triangledown f_1(a) \\ \vdots \\ \triangledown f_m(a)
         \end{pmatrix}$.
\end{corollary}

\begin{statement}
    $\frac{\partial f}{\partial x_k}(a)=\langle \triangledown f(a), e_k\rangle$, то есть $\triangledown f(a)= (\frac{\partial f}{\partial x_1}(a), ..., \frac{\partial  f}{\partial x_n}(a))$.
\end{statement}

\begin{theorem}
    Пусть $f: E\rightarrow \R$, $E\subset \R^n$, $a\in \Int E$. Если все частные производные функции $f$ непрерывны в точке $a$, то $f$  дифференцируема в точке $a$.
\end{theorem}

\begin{definition}
    $f$ \textit{непрерывно дифференцируема в точке $a$}, если $f$ дифференцируема в окрестности точки $a$ и $\|d_xf-d_af\|\underset{x\rightarrow a}{\rightarrow} 0$.
\end{definition}

\begin{theorem}
    Пусть $f: E\rightarrow \R^m$, $E\subset \R^n$, $a\in \Int E$, $f$ дифференцируема в окрестности точки $a$. Тогда $f$ непрерывно дифференцируема в точке $a\Leftrightarrow\pdv{f_i}{x_j}$ непрерывна в точке $a$ $\forall i, j$. 
\end{theorem}


\begin{theorem}
    \textbf{Формула Тейлора с остатком в форме Пеано.}

    Пусть $D\subset \R^n$ открытое, $f\in C^{r}(D)$, $a\in D$. Тогда при $x\rightarrow a$:
    
    $f(x)=\sum\limits_{|k|\leq r}\frac{f^{(k)}(a)}{k!}(x-a)^k+o(\|x-a\|^r)$
\end{theorem}

\begin{theorem}
    \textbf{Необходимое условие экстремума.}

    Пусть $f: E\rightarrow \R$, $E\subset \R^n$, $a\in \Int E$, $a$ – точка экстремума функции $f$. Тогда если существует $\frac{\partial f}{\partial x_k}(a)$, то $\frac{\partial f}{\partial x_k}(a)=0$. В частности, если $f$ дифференцируема в точке $a$, то $\frac{\partial f}{\partial x_1}(a)=...=\frac{\partial f}{\partial x_n}(a)=0$, то есть $\triangledown f(a)=0$.
\end{theorem}

\begin{theorem}
    \textbf{Достаточные условия экстремума.}

    Пусть $f: E\rightarrow \R$, $E\subset \R^n$, $a\in \Int E$, $a$ – стационарная точка, $f$ дважды дифференцируема, $Q(h):=\sum\limits_{i=1, j=1}^n\frac{\partial^2 f}{\partial x_i \partial x_j}(a) h_i h_j$. Тогда:
    \begin{enumerate}
        \item Если $Q$ строго положительно определена, то $a$ точка строгого локального минимума.
        \item Если $Q$ строго отрицательного определена, то $a$ точка строгого локального максимума
        \item Если $a$ точка нестрогого локального минимума, то $Q$ нестрого положительно определена.
        \item Если $a$ точка нестрогого локального максимума, то $Q$ нестрого отрицательно определена.
    \end{enumerate}
\end{theorem}

\begin{theorem}
    \textbf{Метод неопределенных множителей Лагранжа}
    
    Пусть $f: D \rightarrow \R$ непрерывно дифференцируема,  $\Phi: D \rightarrow \R^m$ непрерывно дифференцируема, $a\in D$, $\Phi(a) =0$, $D$ открытое. 

    Если $a$ точка условного экстремума (при условии $\Phi(x) = 0$), то $\triangledown f_{(a)}$, $\triangledown \Phi_1^{(a)}$, ..., $\triangledown \Phi_m^{(a)}$ линейно зависимы.
\end{theorem}

\newpage

\end{document}